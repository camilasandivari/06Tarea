\documentclass[prl,showpacs, onecolumn]{revtex4-1}
\usepackage{graphicx,amssymb,color}
\usepackage[dvipsnames]{xcolor}
%\usepackage[utf8]{inputenc}
\usepackage[spanish]{babel}
\usepackage{pgfgantt}


\begin{document}
\title{FI3104 M\'etodos N\'umericos para la Ciencia e Ingenieria\\ Tarea 6}
\author{Camila Sandivari}
\affiliation{Profesor: Valentino Gonzalez \\ Profesor Auxiliar: Felipe Pesce}
\date{\today}

\begin{abstract}
El presente reporte muestra la resoluci�n de la ecuaci�n de Fisher-KPP, de reacci�n y difusi�n. Para lograr el objetivo se debe discretizar usando m�todos distintos, la parte de la reacci�n se resuelve con un m�todo de Euler expl�cito y la parte de la difusi�n con Cranck-Nickolson. 
\end{abstract}
\maketitle
\paragraph{Procedimiento}
\subparagraph{Parte 1}
Para resolver la ecuaci�n de Fisher-KPP (1) con sus respectivas condiciones iniciales, se implementa un algoritmo que resuelve la parte de difusi�n con m�todo Cranck-Nickolson y la de reacci�n con m�todo Euler expl�cito. 

\begin{equation}
\frac{\partial n}{\partial t} = \gamma \frac{\partial^2n}{\partial x^2} + \mu n - \mu n^2
\end{equation}
\begin{eqnarray}
 n(t, 0) &= 1\\ n(t, 1) &= 0\\ n(0, x) &= e^{-x^2/0.1} 
\end{eqnarray}

Con $n(x,t)$ la densidad de la especie en tiempo , $\mu n$ la tendencia de la especie a crecer indefinidamente, $-\mu n^{2}$ el factor competencia debido al aumento de densidad de poblaci�n, y $\gamma \nabla n$ la tendencia de la especie a dispersarse para encontrar m�s recursos.\\
Los m�todos de resoluci�n se obtienen de resolver la ecuaci�n (2) mediante discretizar de la forma mostrada en (3). Cuando el par�metro a es igual a 0 se tiene el m�todo de Euler expl�cito que resuelve la parte de reacci�n del problema, y cuando a igual a 1 se conoce como m�todo Cranck-Nickolson y resuelve la parte de difusi�n del problema.
\begin{center}
\begin{equation}
\frac{\partial T}{\partial t} = \frac{\partial^2 T}{\partial x^2}
\end{equation}
\end{center}
\begin{equation}
\frac{T^{n+1}_{k} - T^{n}_{k}}{\epsilon} =\\
 \frac{a}{2} \left[ \frac{T^{n+1}_{k+1} - 2T^{n+1}_{k}+T^{n+1}_{k-1}}{h^{2}}\right] +  \frac{2-a}{2} \left[ \frac{T^{n}_{k+1} - 2T^{n}_{k}+T^{n}_{k-1}}{h^{2}}\right]
\end{equation}

El algoritmo implementado busca resolver ecuaciones parab�licas lineales, asumiendo que son de la forma $A\overrightarrow{\varphi}=\overrightarrow{b}$ y buscando los par�metros que la ajustan a los casos particulares de cada problema a resolver.

\subparagraph{Parte 2}
El problema dos es an�logo al anterior pero se busca resolver el siguiente sistema:

\begin{equation}
\frac{\partial n}{\partial t} = \gamma \frac{\partial^2n}{\partial x^2} + \mu ( n - n^3)
\end{equation}
\begin{eqnarray}
n(t, 0) &= 0\\ n(t, 1) &= 0\\ n(0, x) &= \texttt{np.random.uniform(low=-0.3, high=0.3, size=Nx)}
\end{eqnarray}

\paragraph{Resultados}
\subparagraph{Parte 1}

Se obtienen los gr\'aficos para el espacio de fase en la figura 1, y la trayectoria en la figura 2, del oscilador de Van der Pool






\subparagraph{Parte 2 }

Se obtiene para la soluci\'on del atractor de Lorenz, un gr\'afico utilizando las condiciones iniciales $[1,1,1]$ para las posiciones $[x,y,z]$

\newpage
\paragraph{Conclusiones}
El m\'etodo Runge-Kutta tanto en la versi\'on implementada como la existente en librer\'ias nos permite abordar problemas de integraci\'on bastante complejos y observar o visualizar el comportamiento de sistemas en este caso ca\'oticos que pueden ser de inter\'es.
Se obtienen resultados que coinciden con los resultados obtenidos hist\'oricamente en la resoluci\'on de estos sistemas.

\end{document}